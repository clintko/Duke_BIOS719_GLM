\documentclass[]{article}
\usepackage{lmodern}
\usepackage{amssymb,amsmath}
\usepackage{ifxetex,ifluatex}
\usepackage{fixltx2e} % provides \textsubscript
\ifnum 0\ifxetex 1\fi\ifluatex 1\fi=0 % if pdftex
  \usepackage[T1]{fontenc}
  \usepackage[utf8]{inputenc}
\else % if luatex or xelatex
  \ifxetex
    \usepackage{mathspec}
  \else
    \usepackage{fontspec}
  \fi
  \defaultfontfeatures{Ligatures=TeX,Scale=MatchLowercase}
\fi
% use upquote if available, for straight quotes in verbatim environments
\IfFileExists{upquote.sty}{\usepackage{upquote}}{}
% use microtype if available
\IfFileExists{microtype.sty}{%
\usepackage{microtype}
\UseMicrotypeSet[protrusion]{basicmath} % disable protrusion for tt fonts
}{}
\usepackage[margin=1in]{geometry}
\usepackage{hyperref}
\hypersetup{unicode=true,
            pdftitle={BIOS719 HW 03 \textbar{} Iterated Weighted Least Square},
            pdfborder={0 0 0},
            breaklinks=true}
\urlstyle{same}  % don't use monospace font for urls
\usepackage{color}
\usepackage{fancyvrb}
\newcommand{\VerbBar}{|}
\newcommand{\VERB}{\Verb[commandchars=\\\{\}]}
\DefineVerbatimEnvironment{Highlighting}{Verbatim}{commandchars=\\\{\}}
% Add ',fontsize=\small' for more characters per line
\usepackage{framed}
\definecolor{shadecolor}{RGB}{248,248,248}
\newenvironment{Shaded}{\begin{snugshade}}{\end{snugshade}}
\newcommand{\KeywordTok}[1]{\textcolor[rgb]{0.13,0.29,0.53}{\textbf{#1}}}
\newcommand{\DataTypeTok}[1]{\textcolor[rgb]{0.13,0.29,0.53}{#1}}
\newcommand{\DecValTok}[1]{\textcolor[rgb]{0.00,0.00,0.81}{#1}}
\newcommand{\BaseNTok}[1]{\textcolor[rgb]{0.00,0.00,0.81}{#1}}
\newcommand{\FloatTok}[1]{\textcolor[rgb]{0.00,0.00,0.81}{#1}}
\newcommand{\ConstantTok}[1]{\textcolor[rgb]{0.00,0.00,0.00}{#1}}
\newcommand{\CharTok}[1]{\textcolor[rgb]{0.31,0.60,0.02}{#1}}
\newcommand{\SpecialCharTok}[1]{\textcolor[rgb]{0.00,0.00,0.00}{#1}}
\newcommand{\StringTok}[1]{\textcolor[rgb]{0.31,0.60,0.02}{#1}}
\newcommand{\VerbatimStringTok}[1]{\textcolor[rgb]{0.31,0.60,0.02}{#1}}
\newcommand{\SpecialStringTok}[1]{\textcolor[rgb]{0.31,0.60,0.02}{#1}}
\newcommand{\ImportTok}[1]{#1}
\newcommand{\CommentTok}[1]{\textcolor[rgb]{0.56,0.35,0.01}{\textit{#1}}}
\newcommand{\DocumentationTok}[1]{\textcolor[rgb]{0.56,0.35,0.01}{\textbf{\textit{#1}}}}
\newcommand{\AnnotationTok}[1]{\textcolor[rgb]{0.56,0.35,0.01}{\textbf{\textit{#1}}}}
\newcommand{\CommentVarTok}[1]{\textcolor[rgb]{0.56,0.35,0.01}{\textbf{\textit{#1}}}}
\newcommand{\OtherTok}[1]{\textcolor[rgb]{0.56,0.35,0.01}{#1}}
\newcommand{\FunctionTok}[1]{\textcolor[rgb]{0.00,0.00,0.00}{#1}}
\newcommand{\VariableTok}[1]{\textcolor[rgb]{0.00,0.00,0.00}{#1}}
\newcommand{\ControlFlowTok}[1]{\textcolor[rgb]{0.13,0.29,0.53}{\textbf{#1}}}
\newcommand{\OperatorTok}[1]{\textcolor[rgb]{0.81,0.36,0.00}{\textbf{#1}}}
\newcommand{\BuiltInTok}[1]{#1}
\newcommand{\ExtensionTok}[1]{#1}
\newcommand{\PreprocessorTok}[1]{\textcolor[rgb]{0.56,0.35,0.01}{\textit{#1}}}
\newcommand{\AttributeTok}[1]{\textcolor[rgb]{0.77,0.63,0.00}{#1}}
\newcommand{\RegionMarkerTok}[1]{#1}
\newcommand{\InformationTok}[1]{\textcolor[rgb]{0.56,0.35,0.01}{\textbf{\textit{#1}}}}
\newcommand{\WarningTok}[1]{\textcolor[rgb]{0.56,0.35,0.01}{\textbf{\textit{#1}}}}
\newcommand{\AlertTok}[1]{\textcolor[rgb]{0.94,0.16,0.16}{#1}}
\newcommand{\ErrorTok}[1]{\textcolor[rgb]{0.64,0.00,0.00}{\textbf{#1}}}
\newcommand{\NormalTok}[1]{#1}
\usepackage{graphicx,grffile}
\makeatletter
\def\maxwidth{\ifdim\Gin@nat@width>\linewidth\linewidth\else\Gin@nat@width\fi}
\def\maxheight{\ifdim\Gin@nat@height>\textheight\textheight\else\Gin@nat@height\fi}
\makeatother
% Scale images if necessary, so that they will not overflow the page
% margins by default, and it is still possible to overwrite the defaults
% using explicit options in \includegraphics[width, height, ...]{}
\setkeys{Gin}{width=\maxwidth,height=\maxheight,keepaspectratio}
\IfFileExists{parskip.sty}{%
\usepackage{parskip}
}{% else
\setlength{\parindent}{0pt}
\setlength{\parskip}{6pt plus 2pt minus 1pt}
}
\setlength{\emergencystretch}{3em}  % prevent overfull lines
\providecommand{\tightlist}{%
  \setlength{\itemsep}{0pt}\setlength{\parskip}{0pt}}
\setcounter{secnumdepth}{0}
% Redefines (sub)paragraphs to behave more like sections
\ifx\paragraph\undefined\else
\let\oldparagraph\paragraph
\renewcommand{\paragraph}[1]{\oldparagraph{#1}\mbox{}}
\fi
\ifx\subparagraph\undefined\else
\let\oldsubparagraph\subparagraph
\renewcommand{\subparagraph}[1]{\oldsubparagraph{#1}\mbox{}}
\fi

%%% Use protect on footnotes to avoid problems with footnotes in titles
\let\rmarkdownfootnote\footnote%
\def\footnote{\protect\rmarkdownfootnote}

%%% Change title format to be more compact
\usepackage{titling}

% Create subtitle command for use in maketitle
\newcommand{\subtitle}[1]{
  \posttitle{
    \begin{center}\large#1\end{center}
    }
}

\setlength{\droptitle}{-2em}

  \title{BIOS719 HW 03 \textbar{} Iterated Weighted Least Square}
    \pretitle{\vspace{\droptitle}\centering\huge}
  \posttitle{\par}
    \author{}
    \preauthor{}\postauthor{}
    \date{}
    \predate{}\postdate{}
  

\begin{document}
\maketitle

\section{Background}\label{background}

This is the iterated weighted least square of a Poisson regression with
a log link.

\section{Import data}\label{import-data}

Consider example 4.4 with data in Table 4.3 (as in class)

\begin{Shaded}
\begin{Highlighting}[]
\NormalTok{y <-}\StringTok{ }\KeywordTok{c}\NormalTok{( }\DecValTok{2}\NormalTok{,  }\DecValTok{3}\NormalTok{, }\DecValTok{6}\NormalTok{, }\DecValTok{7}\NormalTok{, }\DecValTok{8}\NormalTok{, }\DecValTok{9}\NormalTok{, }\DecValTok{10}\NormalTok{, }\DecValTok{12}\NormalTok{, }\DecValTok{15}\NormalTok{)}
\NormalTok{x <-}\StringTok{ }\KeywordTok{c}\NormalTok{(}\OperatorTok{-}\DecValTok{1}\NormalTok{, }\OperatorTok{-}\DecValTok{1}\NormalTok{, }\DecValTok{0}\NormalTok{, }\DecValTok{0}\NormalTok{, }\DecValTok{0}\NormalTok{, }\DecValTok{0}\NormalTok{,  }\DecValTok{1}\NormalTok{,  }\DecValTok{1}\NormalTok{,  }\DecValTok{1}\NormalTok{)}
\end{Highlighting}
\end{Shaded}

\section{\texorpdfstring{Q1 Fit model
\(\log{E[Y_i]} = \beta_1 + \beta_2 x_i\) with the glm function in
R}{Q1 Fit model \textbackslash{}log\{E{[}Y\_i{]}\} = \textbackslash{}beta\_1 + \textbackslash{}beta\_2 x\_i with the glm function in R}}\label{q1-fit-model-logey_i-beta_1-beta_2-x_i-with-the-glm-function-in-r}

Using the glm function to perform Poisson regression with a log link

\begin{Shaded}
\begin{Highlighting}[]
\NormalTok{### family: Poisson}
\NormalTok{### link  : log}
\NormalTok{res <-}\StringTok{ }\KeywordTok{glm}\NormalTok{(y }\OperatorTok{~}\StringTok{ }\NormalTok{x, }\DataTypeTok{family =} \KeywordTok{poisson}\NormalTok{(}\DataTypeTok{link =} \StringTok{"log"}\NormalTok{))}
\KeywordTok{summary}\NormalTok{(res)}
\end{Highlighting}
\end{Shaded}

\begin{verbatim}
## 
## Call:
## glm(formula = y ~ x, family = poisson(link = "log"))
## 
## Deviance Residuals: 
##     Min       1Q   Median       3Q      Max  
## -0.8472  -0.2601  -0.2137   0.5214   0.8788  
## 
## Coefficients:
##             Estimate Std. Error z value Pr(>|z|)    
## (Intercept)   1.8893     0.1421  13.294  < 2e-16 ***
## x             0.6698     0.1787   3.748 0.000178 ***
## ---
## Signif. codes:  0 '***' 0.001 '**' 0.01 '*' 0.05 '.' 0.1 ' ' 1
## 
## (Dispersion parameter for poisson family taken to be 1)
## 
##     Null deviance: 18.4206  on 8  degrees of freedom
## Residual deviance:  2.9387  on 7  degrees of freedom
## AIC: 41.052
## 
## Number of Fisher Scoring iterations: 4
\end{verbatim}

From the summary, the glm function return the formula\\
Note: \(\beta: \text{estimate (s.e.)}\)

\[\log{E[Y_i]} = 1.89 (0.14) + 0.67 (0.18) x_i\]

\section{Q2 write an iteratively reweighted least squares R function to
solve the problem in
Q1}\label{q2-write-an-iteratively-reweighted-least-squares-r-function-to-solve-the-problem-in-q1}

\subsubsection{Derive the equations}\label{derive-the-equations}

From the lecture or textbook, we know that:
\[z = \eta_i + \frac{1}{\mu_i} (y_i - \mu_i)\]

\[\omega_{ii} = \frac{1}{\text{var}(Y_i)} \Big( \frac{\partial \mu_i}{\partial \eta_i} \Big)^2\]

with the log link and Poisson regression, we can know that:
\[\begin{cases}
g(\mu_i) = \log{\mu_i} = \eta \\
\mu_i = \exp{ \{ \eta_i \}}   \\
\frac{\partial \eta_i}{\partial \mu_i} = \frac{1}{\mu_i} \\
\frac{\partial \mu_i}{\partial \eta_i} = \mu_i \\
\mu_i = \text{var}(Y_i)
\end{cases}\]

substitue what we know to calculate z and W

\[z = \eta_i + \frac{1}{\mu_i} (y_i - \mu_i) = \eta_i + \exp{\{-\mu_i\}} \Big(y_i - \exp{\{\mu_i\}} \Big) = \eta_i + y_i \exp{\{-\mu_i\}} - 1\]

\[\omega_{ii} = \frac{1}{\text{var}(Y_i)} \Big( \frac{\partial \mu_i}{\partial \eta_i} \Big)^2 = \exp{\{-\eta_i\}} \mu_i^2 = \exp{\{-\eta_i\}} \exp{\{\eta_i\}} \exp{\{\eta_i\}} = \exp{\{\eta_i\}}\]

\subsubsection{Implement iteratively reweighted least
squares}\label{implement-iteratively-reweighted-least-squares}

Write function to perform iteratively reweighted least squares

\begin{Shaded}
\begin{Highlighting}[]
\NormalTok{### set helper functions to perform iterative reweighted least squares}
\NormalTok{get_z <-}\StringTok{ }\ControlFlowTok{function}\NormalTok{(y, X, beta, eta)\{}
    \CommentTok{# get next z vector}
    \CommentTok{#======================}
    \KeywordTok{return}\NormalTok{(eta }\OperatorTok{+}\StringTok{ }\NormalTok{y }\OperatorTok{*}\StringTok{ }\KeywordTok{exp}\NormalTok{(}\OperatorTok{-}\NormalTok{eta) }\OperatorTok{-}\StringTok{ }\DecValTok{1}\NormalTok{)}
\NormalTok{\}}

\NormalTok{get_w <-}\StringTok{ }\ControlFlowTok{function}\NormalTok{(y, X, beta, eta)\{}
    \CommentTok{# get next W matrix}
    \CommentTok{#======================}
    \KeywordTok{dim}\NormalTok{(eta) <-}\StringTok{ }\OtherTok{NULL}
    \KeywordTok{return}\NormalTok{(}\KeywordTok{diag}\NormalTok{(}\KeywordTok{exp}\NormalTok{(eta)))}
\NormalTok{\}}

\NormalTok{### Kernel functions of iterative reweighted least squares}
\NormalTok{get_next_beta <-}\StringTok{ }\ControlFlowTok{function}\NormalTok{(y, X, beta, my_get_w, my_get_z)\{}
    \CommentTok{# get next parameters}
    \CommentTok{#======================}
\NormalTok{    eta <-}\StringTok{ }\NormalTok{X }\OperatorTok\StringTok{ }\NormalTok{beta}
\NormalTok{    z   <-}\StringTok{ }\KeywordTok{my_get_z}\NormalTok{(y, X, beta, eta)}
\NormalTok{    W   <-}\StringTok{ }\KeywordTok{my_get_w}\NormalTok{(y, X, beta, eta)}
    \KeywordTok{return}\NormalTok{(}\KeywordTok{solve}\NormalTok{(}\KeywordTok{t}\NormalTok{(X) }\OperatorTok\StringTok{ }\NormalTok{W }\OperatorTok\StringTok{ }\NormalTok{X) }\OperatorTok\StringTok{ }\KeywordTok{t}\NormalTok{(X) }\OperatorTok\StringTok{ }\NormalTok{W }\OperatorTok\StringTok{ }\NormalTok{z)}
\NormalTok{\}}

\NormalTok{iter_reweight_lsq <-}\StringTok{ }\ControlFlowTok{function}\NormalTok{(y, X, beta, my_get_w, my_get_z, }\DataTypeTok{num_iter =} \DecValTok{100}\NormalTok{)\{}
    \CommentTok{# iteratively reweighted least squares}
    \CommentTok{#======================}
\NormalTok{    ### initialization}
\NormalTok{    res <-}\StringTok{ }\KeywordTok{list}\NormalTok{()}
    
\NormalTok{    ### loop to update the beta}
    \ControlFlowTok{for}\NormalTok{(dummy_num }\ControlFlowTok{in} \DecValTok{1}\OperatorTok{:}\DecValTok{50}\NormalTok{)\{}
\NormalTok{        beta <-}\StringTok{ }\KeywordTok{get_next_beta}\NormalTok{(y, X, beta, my_get_w, my_get_z)}
\NormalTok{    \} }\CommentTok{# end for loop}
    
\NormalTok{    ### store results and calculate final eta & covariance matrix of beta}
\NormalTok{    res}\OperatorTok{$}\NormalTok{beta_est <-}\StringTok{ }\NormalTok{beta}
\NormalTok{    eta <-}\StringTok{ }\NormalTok{X }\OperatorTok\StringTok{ }\NormalTok{beta}
\NormalTok{    W <-}\StringTok{ }\KeywordTok{my_get_w}\NormalTok{(y, X, beta, eta)}
\NormalTok{    res}\OperatorTok{$}\NormalTok{beta_cov <-}\StringTok{ }\KeywordTok{solve}\NormalTok{(}\KeywordTok{t}\NormalTok{(X) }\OperatorTok\StringTok{ }\NormalTok{W }\OperatorTok\StringTok{ }\NormalTok{X)}
    
    \KeywordTok{return}\NormalTok{(res)}
\NormalTok{\}}
\end{Highlighting}
\end{Shaded}

perform iterative reweighted least squares of the data

\begin{Shaded}
\begin{Highlighting}[]
\NormalTok{### intialize the first beta}
\NormalTok{beta <-}\StringTok{ }\KeywordTok{c}\NormalTok{(}\DecValTok{7}\NormalTok{, }\DecValTok{5}\NormalTok{)}

\NormalTok{### set up design matrix}
\NormalTok{X <-}\StringTok{ }\KeywordTok{cbind}\NormalTok{(}\KeywordTok{rep}\NormalTok{(}\DecValTok{1}\NormalTok{, }\KeywordTok{length}\NormalTok{(x)), x)}

\NormalTok{### iterated reweighted least squares}
\NormalTok{res <-}\StringTok{ }\KeywordTok{iter_reweight_lsq}\NormalTok{(y, X, beta, get_w, get_z)}
\KeywordTok{print}\NormalTok{(res)}
\end{Highlighting}
\end{Shaded}

\begin{verbatim}
## $beta_est
##        [,1]
##   1.8892720
## x 0.6697856
## 
## $beta_cov
##                         x
##    0.02019584 -0.01419063
## x -0.01419063  0.03192892
\end{verbatim}

\section{Q3 Compare values obtained in (a) and (b) with results in
Q1}\label{q3-compare-values-obtained-in-a-and-b-with-results-in-q1}

Here I have the same values obtained in Q2(a)(b) and Q1

the results of Q1

\begin{Shaded}
\begin{Highlighting}[]
\NormalTok{res <-}\StringTok{ }\KeywordTok{glm}\NormalTok{(y }\OperatorTok{~}\StringTok{ }\NormalTok{x, }\DataTypeTok{family =} \KeywordTok{poisson}\NormalTok{(}\DataTypeTok{link =} \StringTok{"log"}\NormalTok{))}
\KeywordTok{coef}\NormalTok{(}\KeywordTok{summary}\NormalTok{(res))}
\end{Highlighting}
\end{Shaded}

\begin{verbatim}
##              Estimate Std. Error   z value     Pr(>|z|)
## (Intercept) 1.8892720  0.1421120 13.294243 2.499914e-40
## x           0.6697856  0.1786866  3.748381 1.779797e-04
\end{verbatim}

the results from Q2(a) and (b)

\begin{Shaded}
\begin{Highlighting}[]
\KeywordTok{cat}\NormalTok{(}\StringTok{""}\NormalTok{,}
    \StringTok{"Beta estimates}\CharTok{\textbackslash{}n}\StringTok{"}\NormalTok{,}
    \KeywordTok{as.vector}\NormalTok{(res}\OperatorTok{$}\NormalTok{beta_est),}
    \StringTok{"}\CharTok{\textbackslash{}n}\StringTok{----------------------------------}\CharTok{\textbackslash{}n}\StringTok{"}\NormalTok{,}

    \CommentTok{# standard error of the beta }
    \CommentTok{# (the square root of the diaganol of beta covariance)}
    \StringTok{"standard error of Beta estimates}\CharTok{\textbackslash{}n}\StringTok{"}\NormalTok{,}
    \KeywordTok{diag}\NormalTok{(res}\OperatorTok{$}\NormalTok{beta_cov)}\OperatorTok{^}\NormalTok{(}\FloatTok{0.5}\NormalTok{))}
\end{Highlighting}
\end{Shaded}

\begin{verbatim}
##  Beta estimates
##  
## ----------------------------------
##  standard error of Beta estimates
## 
\end{verbatim}


\end{document}
